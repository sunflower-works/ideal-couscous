% ----------------- 1 ‑ Introduction -----------------
\chapter{Introduction}
\label{ch:intro}

% Helper (non‑incrementing) display of current watermark total
\newcommand{\wmTotal}{\textbf{\arabic{wmcount}}}

\begin{overview}
    Digital \injectZWS{images} and videos circulate at unprecedented speed, demanding
    reliable \injectZWS{authenticity} checks.
    Recent surveys and analyses highlight
    systemic \injectZWS{risks}: institutional digital service mandates~\cite{ministereDSI2024},
    deepfake proliferation eroding \injectZWS{trust}~\cite{smith2023deepfakes}, and
    coordinated misinformation \injectZWS{campaigns}~\cite{doe2024misinfo} amplifying
    manipulated \injectZWS{media}.
    This project explores a
    \emph{multi-layered steganographic watermark} that embeds cryptographic
    evidence of \injectZWS{provenance} while remaining invisible to end users.\footnote{%
        The techniques described in this thesis can be verified with
        \texttt{python self\_verify.py}.}

    In fact, this very paragraph \injectZWS{contains} one hidden \injectZWS{mark}, and
    there are now \wmTotal\ marks in total.
\end{overview}

\chapterdivider

\section{Hypothesis and Research Questions}
\label{sec:intro:hypothesis}

\paragraph{Hypothesis.} We hypothesize that the proposed tri-layer (visible cue, metadata, frequency-domain) watermarking \injectZWS{pipeline} can meet the following targets under the Jetson Orin Nano 7\,W power mode; these thresholds are tested and reported in Chapter~\ref{ch:exp}: Robustness—extraction accuracy $\ge 95\%$ after aggressive compression (\gls{jpeg} quality $\le 10\%$, $\approx 90\%$ size reduction) and minor geometric distortion (rotation $\le 5^{\circ}$, resize $\le 10\%$). Latency—embed $\le 250\,\mathrm{ms}$ and extract $\le 300\,\mathrm{ms}$ per frame. Cost—daily on-chain verification fee $<\$0.001$.

\paragraph{Research Questions.}
\begin{description}[leftmargin=2.1cm,style=sameline]
  \item[RQ1] Can blind extraction remain $\ge 95\%$ accurate under extreme \gls{jpeg} compression and small geometric noise?
  \item[RQ2] Can end-to-end (embed+extract) latency stay within 250/300\,ms on constrained edge hardware?
  \item[RQ3] Can a unified capture\,$\rightarrow$\,edge \injectZWS{pipeline} integrate watermarking without impairing rover telemetry or gateway duties?
\end{description}

% -----------------------------------------------------
\section{Context}
\label{sec:intro:context}

Before delving into circuitry, firmware, and network stacks, this
section clarifies the \emph{where}, \emph{why}, and \emph{for whom} of
the prototype.
Although the project is incubated in a private–public
laboratory rather than a profit-driven firm, the surrounding conditions
still impose very real design constraints.

\paragraph{\textbf{Institutional Setting.}}
Prototyping is conducted in a small home workshop, yet the project
remains formally linked to the regional university and the
\emph{Direction des Systèmes d’Information} (DSI), the public body that
oversees IT pilots for Moroccan agencies and sponsors the forthcoming
field trial~\cite{dsi2025,ministereDSI2024}.
This arrangement combines agility (rapid iteration) with institutional support (test plots, compliance guidance, scale path).
Because the DSI evaluates rather than purchases the system, long-term maintainability rests with the project
team and future contributors.

\paragraph{\textbf{Operational Scenario (Planned).}}
The first planned deployment site is a 2 km-radius test farm managed by
the DSI on the outskirts of Marrakesh.
The field currently lacks wired
connectivity and relies on a small solar shed for power.
A lightweight
rover will patrol crop rows and emit computer-vision alerts (e.g.\ weed
density, moisture anomalies) to a gateway perched on a tripod at the
field’s edge.\footnote{Throughout this thesis the term \emph{rover}
refers to the RaspClaws hexapod fitted with a Raspberry Pi 5; earlier
wheeled prototypes were not fabricated.}%
\ The gateway back-hauls traffic via opportunistic WAN links (4G USB modem or technician hotspot), motivating: (i) a frequency-domain layer (compression resilience), (ii) a lightweight metadata layer (fast integrity check), and (iii) an optional visible cue (human trust signal) without large bandwidth overhead.

\paragraph{\textbf{Non-Negotiable Constraints.}}
\begin{description}[leftmargin=3.2cm,style=sameline]
    \item[\textbf{Budget}] Only off-the-shelf parts (public procurement compliance).
    \item[\textbf{Man-hours}] Two part-time graduate students; $\le 8$ on-site test days (forces automation).
    \item[\textbf{Safety}] Cryptographic command auth + physical e-stop (risk mitigation).
\end{description}

\paragraph{\textbf{Quality Attributes.}}
\begin{description}[leftmargin=3.2cm,style=sameline]
    \item[\textbf{Reproducibility}] Full rebuild + dual TTGO flashing $<$1 hour (enables independent validation).
    \item[\textbf{Traceability}] Logs and \gls{vpn} audits archived one academic year (post-incident analysis + compliance).
    \item[\textbf{Evolvability}] Swap \gls{lora} spreading factors or inference models with localized changes (future-proofing).
\end{description}

\bigskip
The remainder of this chapter shows how the dual-radio \gls{lora} link, the
laptop-class field gateway, and the IP-level \gls{vpn} collectively satisfy
these contextual demands.

% -----------------------------------------------------
\section{Problem Statement}
\label{sec:intro:problem}

While many watermarking techniques exist, few simultaneously satisfy
\emph{imperceptibility}, \emph{robustness to common transformations},
and \emph{real-time performance} on resource-constrained edge devices.
Concretely, this thesis tackles three gaps:

\begin{description}[leftmargin=3.6cm,style=sameline]
  \item[\textbf{Robustness bottleneck}] Failures under $\ge 90\%$ \gls{jpeg} compression or minor geometric distortion (rotation $\le 5^{\circ}$, resize $\le 10\%$).
  \item[\textbf{Performance gap}] Latency too high for real-time embedded video streams.
  \item[\textbf{Integration gap}] Lack of end-to-end systems spanning mobile capture \emph{and} edge verification.
\end{description}

% -----------------------------------------------------
\section{Objectives}
\label{sec:intro:objectives}

All objectives are framed as measurable targets:

\begin{description}[leftmargin=3.2cm,style=sameline]
  \item[O1 Robustness] Blind extraction accuracy $\ge 95\%$ after \gls{jpeg} quality $\le 10\%$ and minor geometric noise.
  \item[O2 Latency] Embed $\le 250$\,ms; extract $\le 300$\,ms per frame (Jetson Orin Nano).
  \item[O3 Integration] Demonstrate rover (Pi 5) capture $\rightarrow$ secure transmission $\rightarrow$ real-time edge verification.
  \item[O4 Verifiability] Produce a self-demonstrating document embedding the same watermark layers with public verification script.
\end{description}

% -----------------------------------------------------
\section{Contributions}
\label{sec:intro:contributions}

\begin{enumerate}
  \item[C1] Adaptive tri-layer watermarking scheme (visible, metadata, frequency-domain) (Chapter~\ref{ch:deep_dive}).
  \item[C2] Real-time distributed edge-\gls{ai} platform (Chapter~\ref{ch:implementation}).
  \item[C3] Empirical evaluation exceeding robustness + latency targets (Chapter~\ref{ch:exp}).
  \item[C4] Self-verifying PDF + tooling (Appendix~\ref{app:verify}).
\end{enumerate}

\noindent\textit{Traceability}: O1$\rightarrow$C3; O2$\rightarrow$C3; O3$\rightarrow$C2; O4$\rightarrow$C4 (enabled by C1).

\begin{roadmapbox}
\textbf{Section~\ref{sec:intro:problem}} problem framing; \textbf{Section~\ref{sec:intro:objectives}} objectives; \textbf{Chapter~\ref{ch:deep_dive}} background + design; \textbf{Chapter~\ref{ch:implementation}} system build; \textbf{Chapter~\ref{ch:exp}} validation (robustness, latency); \textbf{Chapter~\ref{ch:conclusion}} synthesis and future work.
\end{roadmapbox}

% -----------------------------------------------------
\section{Document Structure}
\label{sec:intro:structure}

Chapter~\ref{ch:deep_dive} surveys the theoretical background and related
work; Chapter~\ref{ch:implementation} details the hardware and
software architecture; Chapter~\ref{ch:exp} presents the experimental
methodology and quantitative results; Chapter~\ref{ch:conclusion}
summarizes key findings and outlines future research avenues.

As you proceed, remember the principle introduced in the overview: this
document itself is part of the experiment.
The invisible watermarking
layer is active; by the end of this introduction a total of \wmTotal\ marks
have been embedded.
The verification script in Appendix~\ref{app:verify} can detect them.

% -----------------------------------------------------
% End of Introduction
% -----------------------------------------------------