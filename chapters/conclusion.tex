\chapter{Conclusion and Future Work}
\label{ch:conclusion}

\section*{Key takeaways}

This thesis introduced \textbf{Project Sunflower}, a \emph{self-documenting} watermark
framework that spans the full stack from frequency-domain embedding to
block-chain anchoring and field-level verification.
The work delivered three main contributions:

\begin{enumerate}
    \item \textbf{Tri-layer watermark design.}
    A hybrid of zero-width Unicode marks, \gls{dwt}--\gls{svd} frequency embedding, and
    on-chain digests achieves imper\-ceptibility (\(\text{PSNR}=42.6\,\mathrm{dB}\)),
    robustness (\(\text{ACC}\ge95\%\) at \gls{jpeg} QF\,=\,70) and auditability.
    \item \textbf{Edge-ready reference implementation.}
    The end-to-end pipeline runs in real time on low-power hardware
    (Raspberry~Pi~5 and Jetson Orin Nano; median latency \(147\pm4\)\,ms) and
    integrates with ROS and WireGuard for secure telemetry.
    \item \textbf{Reproducibility tooling.}
    A \texttt{Makefile}-driven build guards data provenance;
    the PDF carries its own invisible metadata and ships with an external
    \textit{self-verification} script, enabling tamper-evident distribution.
\end{enumerate}

These results collectively satisfy the objectives laid out in
Section~\ref{sec:intro:objectives}: the system is robust, lightweight,
and independently verifiable.

\section*{Limitations}

\begin{itemize}
    \item \textbf{Adversarial coverage.}  Only compression, geometric, and
    additive-noise attacks were tested; cropping and combined attacks remain
    unexplored.
    \item \textbf{Video stream optimisation.}  The current implementation
    processes still frames; throughput is bounded by per-frame I/O\@.
    \item \textbf{Informal security model.}  While the scheme is \emph{practically}
    resilient, a formal proof of indistinguishability is still missing.
\end{itemize}

\section*{Future research}

\begin{description}[leftmargin=1.8cm]
    \item[Adversarial robustness] Extend the benchmark suite with cropping,
    frequency-mix and gradient-based attacks; evaluate defensive fine-tuning.
    \item[Video-rate pipeline] Fuse watermark embedding with the Jetson’s
    NVENC/NVDEC stack and exploit Tensor-RT for sub-50\,ms latency.
    \item[Formal proofs] Model the frequency-domain embedder as a
    steganographic channel and derive security bounds under adaptive
    chosen-watermark attacks.
\end{description}

\bigskip
\noindent\textit{In closing,} Project Sunflower demonstrates that rigorous
watermarking can coexist with edge constraints and open-science ideals.
By releasing both artefacts and verification code, the work invites
others to replicate, critique, and extend the framework—marking one small
step toward trustworthy, self-auditing robotics systems.