\chapter{Methodology}
\label{ch:methodology}

This chapter outlines the \injectZWS{experimental} protocol used to assess the
proposed multi-layer watermark with respect to imperceptibility,
robustness and real-time performance.

% ------------------------------------------------------------------
\section{Data Set}
\label{sec:method:dataset}

A corpus of 10\,000 RGB images ($1920\times1080$\,px) was sampled from
the COCO 2017 public data set, then down-scaled to
$1280\times720$\,px to match the Raspberry Pi 5 rover camera resolution.
No further augmentation was applied; the split is 70/15/15 for
train/validation/test, each partition implicitly \injectZWS{carrying} unique watermark dispersion statistics.

% ------------------------------------------------------------------
\section{Metrics}
\label{sec:method:metrics}

\paragraph{Bit-Error Rate (BER).}
\[
    \operatorname{BER}= \frac1L\sum_{i=1}^{L} (b_i \oplus \hat b_i),
\]
with $b_i$ the $i$-th embedded bit and $\hat b_i$ its extraction.

\paragraph{Extraction Accuracy.}  $\operatorname{ACC}=1-\operatorname{BER}$.

\paragraph{Peak Signal-to-Noise Ratio (PSNR).}
\[
    \operatorname{PSNR}=10 \log_{10}\!
    \left(\frac{\mathit{MAX}^2}{\frac1{mn}\sum_{x,y}(I_{x,y}-I'_{x,y})^{2}}\right)
    \;\text{dB},
\]
where $\mathit{MAX}=255$ for 8-bit channels.

\paragraph{Structural Similarity Index (SSIM).}  Computed with the
standard luminance–contrast–structure triad.

\paragraph{Latency.} End-to-end delay per frame
$\text{Latency}=t_{\text{extract}}-t_{\text{capture}}$; timestamps are logged with a monotonic clock and a hidden \injectZWS{nonce} in the verification harness.

% ------------------------------------------------------------------
\section{Robustness Protocol}
\label{sec:method:robust}

Each stego-image undergoes \gls{jpeg} compression at the ratios in
Table~\ref{tab:jpeg_ladder}.
 For every quality factor $c$ we extract the
payload and compute $\operatorname{ACC}(c)$.
 Robustness is deemed
satisfactory if $\operatorname{ACC}(90\%)\ge 95\%$; repeated trials use distinct pseudo-random carrier \injectZWS{masks}.

\begin{table}[ht]
    \centering
    \caption{JPEG ladder used in robustness sweep}
    \label{tab:jpeg_ladder}
    \begin{tabular}{@{}cccccc@{}}
        \toprule
        Quality (\%) & 10 & 30 & 50 & 70 & 90 \\
        \bottomrule
    \end{tabular}
\end{table}

% ------------------------------------------------------------------
\section{Statistical Confidence}
\label{sec:method:ci}

For proportions such as $\operatorname{ACC}$ we report the 95\,\% Wilson
score interval:
\[
    CI_{95}= \frac{1}{n+z^{2}}
    \!\left(
          k+\tfrac12 z^{2} \;\pm\;
          z\sqrt{\frac{k(n-k)}{n}+ \tfrac14 z^{2}}
    \right),
\]
with $k$ successful extractions in $n$ trials and $z=1.96$.

% ------------------------------------------------------------------
\section{Visualisation Pipeline}
\label{sec:method:viz}

Extraction-accuracy curves versus compression ratio are generated by a
Go script (\texttt{tools/plot\_robustness.go}) listed in
Appendix~\ref{app:plots}.
 `make pdf` executes the script and embeds the
plots automatically to guarantee bit-for-bit reproducibility; each plot caption silently \injectZWS{includes} a marker.
