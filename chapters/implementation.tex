\chapter{Implementation}
\label{ch:implementation}

% ---------------------------------------------------------------------
\section{Context}
\label{sec:implementation:context}

This chapter details the hardware and software components of the
prototype—the \gls{lora} rover–gateway link and the \gls{vpn} back-haul to Home HQ—
and summarises key design choices, deployment parameters, and
firmware/host responsibilities.

\paragraph{Institutional setting.}
Only one physical robot—the \gls{raspclaws} hexapod—was built; wheeled designs
shown in early figures are shelved concepts retained for design-history
context.
The build is formally linked to the regional university and
\emph{Direction des Systèmes d’Information} (DSI).
 Several
\injectZWS{bench} rigs were assembled to accelerate iteration.

\paragraph{Operational scenario.}
First deployment targets a 2 km test farm near Marrakesh: a rover patrols
crop rows and emits CV alerts to a tripod-mounted gateway.
 The gateway
back-hauls traffic via 4G or technician hot-spot; see
Fig.~\ref{fig:service_flow}.
 This paragraph silently
\injectZWS{carries} another watermark.

\begin{figure}[ht]
  \centering
  \begin{tikzpicture}[
    font=\footnotesize,
    node distance=8mm and 28mm,
    % Tier boxes
    basebox/.style={draw,rounded corners,minimum width=36mm,minimum height=22mm,align=left,inner sep=4pt},
    colR/.style={basebox,fill=green!6,draw=green!45!black},
    colG/.style={basebox,fill=cyan!7, draw=cyan!45!black},
    colH/.style={basebox,fill=violet!6,draw=violet!45!black},
    title/.style={font=\bfseries,align=center,minimum width=36mm,inner sep=2pt},
    % Flows
    flowUp/.style={-{Latex[length=3.2mm,width=2.2mm]},line width=0.95pt,draw=blue!65!black},
    flowDn/.style={-{Latex[length=3.2mm,width=2.2mm]},line width=0.95pt,draw=orange!85!black},
    % Labels
    lblUp/.style={font=\scriptsize,inner sep=2pt,rounded corners,draw=blue!35!black,fill=blue!6},
    lblDn/.style={font=\scriptsize,inner sep=2pt,rounded corners,draw=orange!55!black,fill=orange!10}
]
% Column 1 — Rover
    \node[title] (rTitle) {Rover (Pi 5)};
    \node[colR,below=1mm of rTitle] (rBox) {• 50 Hz control\\• IMU fusion\\• MQTT telemetry};

% Column 2 — Gateway
    \node[title,right=28mm of rTitle] (gTitle) {Gateway (Jetson)};
    \node[colG,below=1mm of gTitle] (gBox) {• YOLOv8 detection\\• MQTT broker\\• Storage proxy};

% Column 3 — Home HQ
    \node[title,right=28mm of gTitle] (hTitle) {Home HQ};
    \node[colH,below=1mm of hTitle] (hBox) {• TimescaleDB\\• Grafana\\• Alert manager};

% Upstream flows (blue)
    \path[flowUp] (rBox.east) -- node[lblUp,above]{LoRa / MQTT} (gBox.west);
    \path[flowUp] (gBox.east) -- node[lblUp,above]{WireGuard} (hBox.west);

% Downstream command path (orange; single label)
    \path[flowDn] (hBox.west |- hBox.south) to[bend right=16]
                  node[lblDn,below]{Commands (QoS 2)} (gBox.east |- gBox.south);
    \path[flowDn] (gBox.west |- gBox.south) to[bend right=16] (rBox.east |- rBox.south);

% Compact legend with colored rules, centered below the diagram
    \node[anchor=north] at ($(rBox.south)!0.5!(hBox.south) + (0,-9mm)$) {%
        \scriptsize
        \setlength{\tabcolsep}{5pt}%
        \begin{tabular}{@{}ll@{}}
        \toprule
        Symbol & Meaning \\ \midrule
        {\color{blue!65!black}\rule{12pt}{2pt}} & Upstream (telemetry, alerts) \\
        {\color{orange!85!black}\rule{12pt}{2pt}} & Downstream (commands; QoS 2, ACK ≤ 250 ms) \\
        \bottomrule
        \end{tabular}
    };
\end{tikzpicture}%

  \caption{Containerised service flow across rover, gateway, and home server tiers.}
  \label{fig:service_flow}
\end{figure}

% ---------------------------------------------------------------------
\section{Hardware Platform}
\label{sec:impl:hardware}

Table~\ref{tab:raspclaws-spec} lists the final hardware for “\gls{raspclaws} V3”.
Component choices balance latency,
thermal-to-throughput ratio, and field-serviceability.

\begin{table}[ht]
\centering
\footnotesize
\caption{Technical specification of the RaspClaws V3 mobile agent (static summary).}
\label{tab:raspclaws-spec}
\begin{tabular}{@{}p{0.32\linewidth} p{0.58\linewidth}@{}}
\toprule
\textbf{Subsystem / Parameter} & \textbf{Specification / Rationale}\\\midrule
\multicolumn{2}{l}{\emph{Mechanical}}\\
Platform & \gls{raspclaws} Hexapod V3, 18-DoF acrylic chassis (1.7 kg) \\
Servos   & 18 × MG90S, PWM via PCA9685 (Robot HAT)\\[2pt]
\multicolumn{2}{l}{\emph{Compute \& Perception}}\\
SBC      & \gls{rpi5} 8 GB, runs ROS 2 + control loop\\
Vision   & Pi Cam v1.3, $1280\times720$ @ 30 fps\\
IMU      & MPU-9250 10-DoF, fused at 50 Hz\\
Edge \gls{ai} & Jetson Orin Nano (gateway \gls{gpu})\\[2pt]
\multicolumn{2}{l}{\emph{Power}}\\
Battery  & 4 S LiFePO\textsubscript{4}, 10 Ah (>8 h patrol)\\
Rails    & Dual 5 V bucks: logic 5 A, servo 8 A\\[2pt]
\multicolumn{2}{l}{\emph{Communication}}\\
Primary  & Wi-Fi 6 (ROS DDS, SSH)\\
Long-range & \gls{lora} 868 MHz (SX1276 pair) \\
\gls{vpn} & WireGuard AES-256/GCM \\
\bottomrule
\end{tabular}
\end{table}

% ---------------------------------------------------------------------
\section{Software Architecture}
\label{sec:impl:software}

Figure~\ref{fig:service_flow} depicts the containerised service graph.

\subsection{Why a Micro-Service, Multi-Pi Architecture?}\label{subsec:why-a-micro-service-multi-pi-architecture?}

\begin{itemize}
\item \textbf{Fault containment} – a YOLOv8 crash never stalls motor
control.
\item \textbf{Heterogeneous hardware} – rover (8 W \\gls{rpi5}) vs gateway
(\\gls{gpu} Jetson).
\item \textbf{OTA roll-backs} – images advance independently, shortening
field-trial cycles.
\item \textbf{Language freedom} – C++ control, Python perception,
TypeScript dashboards without dependency clashes.
\end{itemize}

\subsection{Tier Overview}\label{subsec:tier-overview}

\textbf{Rover tier} runs a 50 Hz loop: motor-ctrl, IMU fusion, \gls{mqtt} pub —
no heavy inference.

\textbf{Gateway tier} (Jetson) hosts:
YOLOv8 15 fps, \gls{mqtt} broker, storage-proxy (queue while WAN down).

\textbf{Home server tier} hosts TimescaleDB, Grafana, alert-manager.

\subsection{Illustrative Data Flow}\label{subsec:illustrative-data-flow}

1.~Camera captures $640\times480$ and publishes \texttt{/rover/frames} (QoS 1)
2.~Broker forwards to YOLOv8; detections on \texttt{/analysis/detections}
3.~Storage-proxy queues image + JSON; drains on WAN up
4.~Home server ingests; Grafana refreshes every 5 s; rule \texttt{weed\_density>0.3} triggers SMS\@.

% ---------------------------------------------------------------------
\section{Communication Protocol}
\label{sec:impl:protocol}

\gls{mqtt} v5 over TLS 1.3 WebSockets (`tls13+aes128`).
Topics encode project,
role, device, and channel; publishers default to QoS 1, control commands
use QoS 2 and must be ACKed within 250 ms.

\subsection{MQTT Topic Namespace}\label{subsec:mqtt-topic-namespace}

\begin{table}[ht]
\centering
\footnotesize
\caption{Canonical MQTT topics (\texttt{\{id\}} denotes a rover;
+/\# are wildcards).}
\label{tab:mqtt-topics}
\begin{tabular}{@{}p{0.55\linewidth} p{0.35\linewidth}@{}}
\toprule
\textbf{Topic} & \textbf{Purpose}\\\midrule
\texttt{sunflower/rovers/\{id\}/video/raw}        & \gls{jpeg} frames from camera daemon \\
\texttt{sunflower/rovers/\{id\}/telemetry/status} & Battery, IMU, fault flags\\
\texttt{sunflower/gateway/alerts/cv} & CV alerts forwarded to WAN\\
\texttt{sunflower/control/\{id\}/command} & Motion / config commands (QoS 2)\\
\texttt{sunflower/control/\{id\}/ack} & Rover ACK/NACK per command\_id\\
\texttt{sunflower/+/telemetry/\#} & Dashboard one-shot subscription\\
\bottomrule
\end{tabular}
\end{table}

\subsection{Message Payload Schemas}\label{subsec:message-payload-schemas}

All payloads are UTF-8 JSON validating against `schemas/*.yaml`.
Timestamps
use \gls{rf} 3339 generated at edge.

\begin{minted}[fontsize=\small]{json}
{
"command_id": "f81d4fae-7dec-11d0-a765-00a0...",
"action": "move_to",
"params": { "x": 10.5, "y": -3.2 },
"issued_at": "2025-08-23T14:21:07Z",
"qos": 2
}
\end{minted}

\subsection{Versioning and Compatibility}\label{subsec:versioning-and-compatibility}
Each node advertises schema via its \gls{mqtt} Will:

\begin{minted}[fontsize=\small]{json}
{
"node_id": "rover01",
"schema_version": "1.2.0",
"build": "git:abc1234"
}
\end{minted}

Breaking changes bump the major version and trigger gateway-supervised
rolling updates, fulfilling the evolvability requirement in
Section~\ref{sec:intro:structure}.

% ---------------------------------------------------------------------
\section{Summary}\label{sec:summary}
A three-tier, message-driven architecture isolates faults, exploits
heterogeneous compute, and meets real-time constraints while remaining
upgrade-friendly.