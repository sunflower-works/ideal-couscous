\chapter{Implementation}
\label{ch:implementation}

% ---------------------------------------------------------------------
\section{Context}
\label{sec:implementation:context}

This chapter details the \injectZWS{hardware} and \injectZWS{software} components of the
prototype—the \gls{lora} rover–gateway link and the \gls{vpn} back-haul to Home HQ—
and summarises key design \injectZWS{choices}, deployment \injectZWS{parameters}, and
firmware/host \injectZWS{responsibilities}.

\paragraph{Institutional setting.}
Only one physical \injectZWS{robot}—the \gls{raspclaws} hexapod—was built; wheeled designs
shown in early figures are shelved concepts retained for design-history
context.
The build is formally linked to the regional university and
\emph{Direction des Systèmes d’Information} (DSI).
 Several
\injectZWS{bench} rigs were assembled to accelerate \injectZWS{iteration}.

\paragraph{Operational scenario.}
First deployment targets a 2 km test farm near Marrakesh: a \injectZWS{rover} patrols
crop rows and emits CV alerts to a tripod-mounted gateway.
 The gateway
back-hauls traffic via 4G or technician hot-spot; see
Fig.~\ref{fig:service_flow}.
 This paragraph silently
\injectZWS{carries} another watermark.

\begin{figure}[ht]
  \centering
  % Detailed end-to-end service flow architecture diagram
\begin{tikzpicture}[
  font=\small,
  node distance=8mm and 12mm,
  box/.style={draw,rounded corners,fill=black!2,minimum width=38mm,inner sep=6pt,align=center},
  circ/.style={draw,circle,minimum width=30mm,inner sep=6pt,fill=white,align=center},
  link/.style={-latex,thick},
  optlink/.style={link,dashed},
  label/.style={font=\scriptsize,fill=gray!20,inner sep=1pt},
]

% ----- LAN / VPN block ------------------------------------------------
\node[box,minimum height=80mm](lan){IP / Wi-Fi / Ethernet};
  \node[circ,above right=of lan.north west,xshift=18mm,yshift=-16mm]         (home){Home Pi\\Central Server};
  \node[circ,below=22mm of home]                                             (cmd){Portable Pi\\Command Center};

  \path[link] (cmd) edge node[label]{\gls{vpn} / SSH} (home);

% ----- WireGuard pair -------------------------------------------------
\node[circ,fill=orange!20,right=28mm of cmd] (wgSrv){WireGuard Srv\\10.81.66.1};
\node[circ,fill=orange!20,below=of wgSrv]    (wgCli){WireGuard Cli\\10.81.66.7};
\path[link] (wgSrv) edge node[label]{WireGuard tunnel} (wgCli);

\path[link] (cmd) edge[out=0,in=180] node[label]{\gls{mqtt} cmd/rover/} (wgCli);
\path[link] (cmd) edge[bend right=10] node[label]{\gls{mqtt} alert/rover/} (wgSrv);

% ----- Mobile field centre block -------------------------------------
\node[box,below=8mm of wgCli,minimum width=55mm,minimum height=45mm] (mob) {MOBILE\\ Command Center};
  \node[box,fill=white,above=2mm of mob] (laptop)
        {OpenSUSE Laptop “Pluton”\\[-1pt]• mqtt Edge Relay\\• Dashboard UI\\• Log Capture};
  \node[box,fill=white,below=2mm of mob] (piZero)
        {Pi Zero 2 W “$n$-node”\\LoRa HAT};

\path[link] (wgCli) edge node[label]{\gls{mqtt} telemetry/rover/} (mob);
\path[link] (piZero) edge node[label]{UDP telemetry} (mob);
\path[link] (piZero) edge[bend left=10] node[label]{UDP 7E \& pkt} (laptop);
\path[link] (laptop) edge node[label]{USB-ECM 192.168.13.x} (mob);

% ----- LoRa base-station stack ---------------------------------------
\node[box,below=of mob,minimum height=55mm] (loraBase){LoRa 868 MHz};
  \node[box,fill=white,above=0 of loraBase] (ttgoUSB){TTGO LoRa32\\USB-UART};
  \node[box,fill=white,below=0 of loraBase] (ttgoSPI){TTGO LoRa32\\SPI/UART};
  \path[link] (ttgoUSB) -- (ttgoSPI);
  \node[label,above left=-1mm of loraBase.north] {USB};

% optional extra antenna
\node[box,fill=white,right=18mm of loraBase] (roof){TTGO LoRa32\\Roof Antenna};
\path[optlink] (roof.west) edge[bend left=10] (loraBase);

% ----- Rover block ----------------------------------------------------
\node[box,left=28mm of loraBase,minimum height=55mm] (rover){Pi Rover};
  \node[circ,fill=white] at (rover) {Rover Pi};
\path[link] (rover.east) edge node[label]{UART/SPI} (loraBase.west);

% camera → Pi → Jetson stack
\node[box,above left=of cmd,xshift=-10mm] (jetson)
    {Jetson Orin Nano\\• YOLO-v8 Detect\\• Watermark + Sign\\• Publishes alerts → mqtt};
\path[link] (jetson) edge[bend left=12] (cmd);
\node[label,midway,sloped,above] at ($(jetson)!0.55!(cmd)$) {\gls{mqtt} alert/rover/};

\node[box,below=14mm of jetson](pi5)
    {Raspberry Pi 5\\• H.265 Encode\\• RTSP Src};
\node[label,below=1mm of pi5] (piLab) {RTSP (720p)};
\path[link] (pi5.south) edge[bend right=20] (cmd.west);

% CSI camera icon
\node[box,above=of pi5](cam){CSI Camera};
\path[link] (cam) -- (pi5);

\end{tikzpicture}

  \caption{Containerised service flow across rover, gateway, and home server tiers.}
  \label{fig:service_flow}
\end{figure}

% ---------------------------------------------------------------------
\section{Hardware Platform}
\label{sec:impl:hardware}

Table~\ref{tab:raspclaws-spec} lists the final hardware for “\gls{raspclaws} V3”.
Component choices balance latency,
thermal-to-throughput ratio, and field-serviceability.

\begin{table}[ht]
\centering
\footnotesize
\caption{Technical specification of the RaspClaws V3 mobile agent (static summary).}
\label{tab:raspclaws-spec}
\begin{tabular}{@{}p{0.32\linewidth} p{0.58\linewidth}@{}}
\toprule
\textbf{Subsystem / Parameter} & \textbf{Specification / Rationale}\\\midrule
\multicolumn{2}{l}{\emph{Mechanical}}\\
Platform & \gls{raspclaws} Hexapod V3, 18-DoF acrylic chassis (1.7 kg) \\
Servos   & 18 × MG90S, PWM via PCA9685 (Robot HAT)\\[2pt]
\multicolumn{2}{l}{\emph{Compute \& Perception}}\\
SBC      & \gls{rpi5} 8 GB, runs \gls{ros} 2 + control loop\\
Vision   & Pi Cam v3, $1280\times720$ @ 30 fps\\
IMU      & MPU-9250 10-DoF, fused at 50 Hz\\
Edge \gls{ai} & Jetson Orin Nano (gateway \gls{gpu})\\[2pt]
\multicolumn{2}{l}{\emph{Power}}\\
Battery  & 4 S LiFePO\textsubscript{4}, 10 Ah (>8 h patrol)\\
Rails    & Dual 5 V bucks: logic 5 A, servo 8 A\\[2pt]
\multicolumn{2}{l}{\emph{Communication}}\\
Primary  & Wi-Fi 6 (\gls{ros} DDS, SSH)\\
Long-range & \gls{lora} 868 MHz (SX1276 pair) \\
\gls{vpn} & WireGuard AES-256/GCM \\
\bottomrule
\end{tabular}
\end{table}

% ---------------------------------------------------------------------
\section{Software Architecture}
\label{sec:impl:software}

Figure~\ref{fig:service_flow} depicts the containerised service graph.

\subsection{Why a Micro-Service, Multi-Pi Architecture?}\label{subsec:why-a-micro-service-multi-pi-architecture?}

\begin{itemize}
\item \textbf{Fault containment} – a YOLOv8 crash never stalls motor
control.
\item \textbf{Heterogeneous hardware} – rover (8 W \\gls{rpi5}) vs gateway
(\\gls{gpu} Jetson).
\item \textbf{OTA roll-backs} – images advance independently, shortening
field-trial cycles.
\item \textbf{Language freedom} – C++ control, Python perception,
TypeScript dashboards without dependency clashes.
\end{itemize}

\subsection{Tier Overview}\label{subsec:tier-overview}

\textbf{Rover tier} runs a 50 Hz loop: motor-ctrl, IMU fusion, \gls{mqtt} pub —
no heavy inference.

\textbf{Gateway tier} (Jetson) hosts:
YOLOv8 15 fps, \gls{mqtt} broker, storage-proxy (queue while WAN down).

\textbf{Home server tier} hosts TimescaleDB, Grafana, alert-manager.

\subsection{Illustrative Data Flow}\label{subsec:illustrative-data-flow}

1.~Camera captures $640\times480$ and publishes \texttt{/rover/frames} (QoS 1)
2.~Broker forwards to YOLOv8; detections on \texttt{/analysis/detections}
3.~Storage-proxy queues image + JSON; drains on WAN up
4.~Home server ingests; Grafana refreshes every 5 s; rule \texttt{weed\_density>0.3} triggers SMS\@.

% ---------------------------------------------------------------------
\section{Communication Protocol}
\label{sec:impl:protocol}

\gls{mqtt} v5 over TLS 1.3 WebSockets (`tls13+aes128`).
Topics encode project,
role, device, and channel; publishers default to QoS 1, control commands
use QoS 2 and must be ACKed within 250 ms.

\subsection{MQTT Topic Namespace}\label{subsec:mqtt-topic-namespace}

\begin{table}[ht]
\centering
\footnotesize
\caption{Canonical MQTT topics (\texttt{\{id\}} denotes a rover;
+/\# are wildcards).}
\label{tab:mqtt-topics}
\begin{tabular}{@{}p{0.55\linewidth} p{0.35\linewidth}@{}}
\toprule
\textbf{Topic} & \textbf{Purpose}\\\midrule
\texttt{sunflower/rovers/\{id\}/video/raw}        & \gls{jpeg} frames from camera daemon \\
\texttt{sunflower/rovers/\{id\}/telemetry/status} & Battery, IMU, fault flags\\
\texttt{sunflower/gateway/alerts/cv} & CV alerts forwarded to WAN\\
\texttt{sunflower/control/\{id\}/command} & Motion / config commands (QoS 2)\\
\texttt{sunflower/control/\{id\}/ack} & Rover ACK/NACK per command\_id\\
\texttt{sunflower/+/telemetry/\#} & Dashboard one-shot subscription\\
\bottomrule
\end{tabular}
\end{table}

\subsection{Message Payload Schemas}\label{subsec:message-payload-schemas}

All payloads are UTF-8 JSON validating against `schemas/*.yaml`.
Timestamps
use \gls{rf} 3339 generated at edge.

\begin{minted}[fontsize=\small]{json}
{
"command_id": "f81d4fae-7dec-11d0-a765-00a0...",
"action": "move_to",
"params": { "x": 10.5, "y": -3.2 },
"issued_at": "2025-08-23T14:21:07Z",
"qos": 2
}
\end{minted}

\subsection{Versioning and Compatibility}\label{subsec:versioning-and-compatibility}
Each node advertises schema via its \gls{mqtt} Will:

\begin{minted}[fontsize=\small]{json}
{
"node_id": "rover01",
"schema_version": "1.2.0",
"build": "git:abc1234"
}
\end{minted}

Breaking changes bump the major version and trigger gateway-supervised
rolling updates, fulfilling the evolvability requirement in
Section~\ref{sec:intro:structure}.

% ---------------------------------------------------------------------
\section{Summary}\label{sec:summary}
A three-tier, message-driven architecture isolates faults, exploits
heterogeneous compute, and meets real-time constraints while remaining
upgrade-friendly.